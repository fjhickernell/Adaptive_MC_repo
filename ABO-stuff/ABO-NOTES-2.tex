\documentclass{article}
\usepackage{amsmath,amsfonts,amssymb}

\newcommand{\fudge}{\mathfrak{C}}
\newcommand{\real}{\mathbb{R}}

\begin{document}

I like how the sandwiching worked out.
Here are some thoughts:

\begin{enumerate}
\item 
Section 5 looks cramped.  I think it should be cut
out and then later become a paper on its own. As it
stands it might get overlooked. Or somebody might
want to implement it but then the paper won't have enough
support. Also, a longer paper with enough supporting
detail won't be novel enough for a good journal.
\begin{enumerate}
\item The followup paper should have a
title to convey the ``fixed relative
error'' idea so it won't get lost.
\item
There is a lot of careful detail involved
in setting up the hybrid of absolute and relative error.
I think that is too much to squeeze into a page or so.
Most people will skim it. It should be a section in 
a larger paper.
\item
Then there should be extensive
numerical evaluation and some thought on how best
to pick the infinite sequence of error tolerances. 
\item
The infinite sequence
calls to mind sequential analysis and processes that
exit either upper or lower boundaries. Also, it starts
to look reminiscent of Chow and Robins (infinite number
of `looks').
\item
The algorithm should be moved out of the theorem
statement.
\end{enumerate}
\item
The paper is getting long. So my earlier remark on Chow
and Robbins and related papers can be abbreviated.
How about
\begin{quotation}
Chow and Robbins (1965) develop a sequential sampling
fixed width confidence interval procedure for the mean.
Their procedure attains the desired coverage level
in the limit as $\varepsilon\to 0$.  But it does not provide
coverage gaurantees for fixed $\varepsilon>0$.
\end{quotation}
We don't necessarily have to mention all the later papers.
\item
p2 CLT provides confidence interval\\
CLT provides a confidence interval
\item
p2
inequalities Hoeffding's\\
inequalities such as Hoeffding's
\item
at (4) we should say something for the $-3$ crowd
\item
p2 below (4) has: the the 
\item
p2
``Firstly'' and  ``Second of all,''\\
Firstly and Second,\\
(or) First and Second,
\item
p3 (maybe)\\
some $d$-variate function $f$\\
some function $f$\\
{}[the domain of $f$ is already
given as $\real^d$]
(I'm less than sure which way this should
go; sometimes redundancy helps.)
\item
p3
practical error estimation for these methods remains a challenge\\
these methods do not provide fixed-width confidence intervals.\\
(Perhaps separate QMC from RQMC.  QMC has no practical error
estimates. RQMC does, but they're not fixed-width.)
\item
p4
I've not previously seen a convention for taking $\gamma=0$
and $\kappa = -2$ when $\sigma = 0$. Do we need $\sigma=0$?
\item
p4 after first display: $sigma$\\
$\sigma$
\item
p4 Slutsky's theorem is in Lehmann and Romano (2005)
page 433.
\begin{verbatim}                  
@book{lehm:roma:2005,
 author = {E. L. Lehmann and J. P. Romano},
 year   = 2005,
 title  = {Testing Statistical Hypotheses},
 publisher = {Springer},
 edition = {3rd},
 address = {New York}
}
\end{verbatim}
\item
p5 Good catch. I had the wrong Hall paper.  The right one is
\begin{verbatim}
@article{hall:1988,
author  = {P. Hall},
year    = 1988,
title   = {Theoretical comparisons of bootstrap confidence intervals},
journal = {The Annals of Statistics},
volume  = 16,
number  = 3,
pages   = {927--953}
}
\end{verbatim}
The coverage errors are on page 948. The usual
intervals correspond to $\pi_{\rm{STUD}}$.
\item
Depending on your appetite for more notation fiddling,
it occurs to me that we might want to
write $n_\sigma$ and $n_\mu$ 
instead of $n_\sigma$ and $n$. 
\item
(17) needs a period (not comma).
\item
(19) has a new wrinkle: making $n\ge n_\sigma$.
Was that condition enforced in the simulations?
I cannot tell but suspect it is not used in 
Section 3.4 on the cost of the algorithm.
I think it is reasonable advice but we should
be sure to spell out when it is used and when it is not.
\item
Theorem 5 statement:\\
fixed width confidence interval\\
fixed width confidence interval condition
\item
Remark 1:\\
should that be $\sigma \le \varepsilon\sqrt{\alpha n_\sigma}$?
(instead of $\sigma^2$)
\item
p9 top:\\
one popular case is occurs when $Y$ is a $d$-variate function\\
one frequently encountered case has $Y$ a $d$-variate function\\
In this case $\mu$\\
Then $\mu$
\item
Equation (22): as mentioned above it is
not clear that this enforces $n\ge n_\sigma$.
The $n_\sigma$ clause is in the definition of $N_\mu$
but I don't see how that clause gets honored
in the displayed math at the bottom of page 9.
That might be the same derivation as before the
clause was added. Then it was ok to just
use $N_{\text{BE}}$ because it was inside a $\min()$.
Now it is also inside a $\max()$. Please double check. 
\item
Figure 1's caption does not describe the many curves
that appear. It should therefore say something about
how those curves are defined in the surrounding text
for the benefit of those who read figures first and
the article second. 
Their full description is too bulky to go into the
caption.
\item
p11 end of Section 3.  Add some examples. A sentence
like
``For example with 
$\fudge = 1.5$ and 
$\tilde \kappa = 10, 100, 1000$
we get $n_\sigma = $ xxx, xxx, and xxx respectively.''
\item
Somewhere we should have a comment like the following:
\begin{quote}
In certain cases our procedure multiplies the
computational cost by a large factor such as $4$ or
$10$ or even $100$ compared to what one might
spend with a known value of $\sigma$. 
While this seems inefficient, it is
well to remember that the total elapsed time may still be
well below one second.
\end{quote}
I leave it to you to put in more appropriate numbers
perhaps after seeing how the finance example comes out.
The comment might belong in the introduction or the
conclusions. It came to mind when I was reading Section 3
though that is probably not where it belongs.
\item
Section 4 title contains ``Algorighm''
\item
p11 has: the the absolute error
\item
The second full sentence on p12 has a grammatical problem.
\item
$131\,072$\\
$131{,}072$
\item
Figure 3: the vertical dashed line and the
horizontal Failure line are prominent and important.
But it is not clear from the caption how they are
defined. Something should be said.
\item
As mentioned above, please double check whether $n\ge n_\sigma$
was enforced for the numerical examples.
\item
As described above I think this section should
expand into its own paper.  I think there are a few 
things to tweak
\begin{enumerate}
\item p14
form of this criterion would be\\
form of this criterion is
\item p15 one must
have $(1-\theta)+\theta|\mu|\ne 0$.
Sometimes it is ok to write ratio criteria
$A/B\le\epsilon$ as $A\le B\epsilon$. But
then you have to watch like a hawk/lawyer to see
whether $\le $ or $<$ is correct, both inside
and outside the probability function.
\item
p15 The sentence containing ``happy'' is not grammatical.
\item
p16 It then follows then by\\
It then follows by
\end{enumerate}
\item
In the discussion where we point out that
non-convexity absolves us
from those theorems on adaptive methods
we might also mention that it also lets
us escape a Bahadur and Savage condition.
\end{enumerate}


\end{document}
