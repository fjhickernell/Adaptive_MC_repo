\documentclass{article}
\usepackage{amsmath,amsfonts,amssymb}

\newcommand{\e}{\mathbb{E}}

\begin{document}

\cite{serf:wack:1976}
consider sequential confidence intervals
for the mean (alternatively for the median)
in parametric distributions, symmetric
about their center point.  The symmetry
condition is not suitable for general
purpose Monte Carlo applications.

\cite{chow:robb:1965} develop a sequential
confidence interval for the mean.
They begin with a sample $X_1,\dots,X_m$ for $m\ge 2$
and then continue sampling until
$$N = \min\{ n\ge m \mid n\ge 
\Phi^{-1}(1-\alpha/2)^2(s_n^2+1/n)/\varepsilon^2\},$$
where $s_n^2 = (n-1)^{-1}\sum_{i=1}^n(X_i-\bar X)^2$.
The confidence interval that they return is of
the form $\bar X \pm \varepsilon$
where $\bar X = (1/n)\sum_{i=1}^nX_i$.
They show that the coverage level approaches
$1-\alpha$ in the limit as $\varepsilon\to 0$.
The coverage error is $O(\epsilon^p)$ for
some $0<p<1/2$. A fixed sample size of
$N^* = \lceil \Phi^{-1}(1-\alpha/2)^2\sigma^2/\varepsilon^2\rceil$
would ordinarily be required to get the desired 
coverage if we knew $\sigma$.
The Chow and Robbins estimate satisfies $N/N^*\to 1$
in expectation and almost surely as $\varepsilon\to 0$.

\cite{mukh:datt:1996} give a procedure similar
to Chow and Robbins' one that
reduces the coverage is at least $1-\alpha+O(\varepsilon^2)$,
at the expense of requiring $\e(|X|^6)<\infty$
and for which $N/N^*\to k>1$.

Our approach by contrast is non-asymptotic. We
consider a fixed level $\varepsilon>0$ and
find an interval of that width with the
desired coverage, so long as the kurtosis is
below a bound.

Our procedure is a two-stage procedure
rather than a fully sequential one.  In
this it is similar to the method of
\cite{stei:1945,stei:1949}, except that
the latter requires normally distributed
data.


\bibliographystyle{plain}
\bibliography{fixwidth}
\end{document}
